% arara: pdflatex: {files: [MathSACpr2014]}
% !arara: indent: {overwrite: yes}
\chapter{Facilities and Support}
\section{Describe how classroom space, classroom technology, laboratory space and equipment impact student success.}

Over the past few years, efforts by the college to create classrooms containing the same basic equipment has helped tremendously with consistency issues.  The nearly universal presence of classroom podiums with attendant AV devices is considerably useful.  For example, most instructors use computer-based calculator emulators when instructing their students on calculator use - this allows explicit keystroking examples to be demonstrated that were not possible before the podiums appeared.  The document cameras found in most classrooms are used by most mathematics instructors.     Having an instructor computer with internet access has also been a great help as instructors have access to a wide variety of tools to engage students, as well as a source for quick answers when unusual questions arise.  

Several classrooms on the Sylvania campus have Starboards  or Smart Boards integrated with their AV systems.  Many mathematics instructors use these tools as their primary presentation vehicles.  Documents can be preloaded into the software and the screens allow instructors to write their work directly onto the document.  Among other things, this makes it easy to save the work into pdf files that can be accessed by students outside of class.  This equipment is not used as much on the other campuses, but there are instructors on other campuses that say they would use them if they were widely available on their campus.

A few instructors have begun creating lessons with LiveScribe technology.  The technology allows the instructor to make an audio/visual record of their lecture without a computer or third person recording device.  The technology allows, among other things, instructors to post a `live copy' of their actual class lecture online.  The students  do not simply see a static copy of the notes that were written;  the students see the notes emerge as they were being written and they hear the words that were spoken while they were written.  The use of LiveScribe technology is strongly supported by Disability Services, and for that reason alone continued experimentation with its use is strongly encouraged.

Despite all of the improvements that have been made in classrooms over the past few years, there still are some serious issues.

Rooms are assigned randomly, which often leads to mathematics classes being scheduled in rooms that are not appropriate for a math class. For example,  scheduling a math class in a room with individual student desks creates a lot of problems; many instructors have students take notes, refer to their text, and use their calculator all at the same time and there simply is not enough room on the individual desktops to keep all of that material in place.  More significantly,  this furniture is especially ill-suited for group work.  Not only does the movement of desks and sharing of work exacerbate the material's issue (materials frequently falling off the desks), students simply cannot share their work in the efficient way that work can be shared when they are gathered about tables.  It would be helpful if all non-computer-based math classes could be scheduled in rooms with tables.

Another problem relates to an inadequate number of computerized classrooms and insufficient space in many of the existing computerized classroom.  Both of these shortages has greatly increased due to Bond-related construction.  Several sections of MTH 243 and MTH 244 (statistics courses), which are normally taught in computerized classrooms, have been scheduled in regular classrooms.  Many of the statistics courses that have been scheduled in computerized classrooms have been scheduled in rooms that seat only 28, 24, or even 20 students.  When possible, we generally limit our class capacities at 34 or 35.  Needless to say, running multiple sections of classes in rooms well below those capacities creates many problems.  This is especially  problematic for student success, as it hinders students'  ability to register due to undersized classrooms.

Finally, the computerized classrooms could be configured in such a way that maximizes potential for meaningful student engagement and minimizes potential for students to get off course due to internet access.  We believe that all computerized classrooms need to come equipped with software that allows the instructor control of the student computers such as LanSchool Classroom Management Software.   The need for this technology is dire; it will reduce or eliminate students being off task when using computers, and it will allow another avenue to facilitate instruction as the instructor will be able to `see' any student computer and `interact' with any student computer.  It can also be used to solicit student feedback in an amomynous manner.  The gathering of anonymous feedback can frequently provide a better gauge of the general level of understanding than activities such as the traditional showing of hands.

\subsection{Recommendations}
All mathematics classes should be scheduled in rooms that are either computerized (upon request) or have multi-person tables (as opposed to individual desks).

All computerized classrooms should have at least 30, if not 34, individual work stations.

An adequate number of classrooms on all campus should be equipped with Smartboards so that all instructors who want access to the technology can teach every one of their classes in rooms equipped with the technology.

The computer image for all computerized classrooms should include software that allows the instructor computer complete and direct access to each student computer. 

\section{Describe how students are using the library or other outside-the-classroom information resources.  }
In order to research how students are using the library and other outside-the-classroom resources, we conducted a stratified sampling method survey of 976 on-campus students and 291 online students; the participants were chosen in a random manner.  We gave scantron surveys to the on-campus students and used SurveyMonkey for the online students. We found that students are generally knowledgeable about library resources and other outside-the-classroom resources.  The complete survey, together with its results, is given in \vref{app:sec:resourcesurvey}; we have summarized our comments below in relation to each question that we asked.

\begin{enumerate}[label=Q\arabic*.,font=\bf]
  \item Not surprisingly, library resources and other campus-based resources are used more frequently by our on-campus students than by our online students. This could be due to less frequent visits to campus for online students and/or online students already having similar resources available to them via the internet. 
  \item We found that nearly 70\% of instructors include resource information in their syllabi.  This figure was consistent regardless of the level of the class (DE/transfer level) or the employment status of the instructor (full/part-time).

    We found that a majority of our instructors are using online resources to connect with students. Online communication between students and instructors is conducted across many platforms such as instructor websites, Desire2Learn, MyPCC, online graphing applications, and online homework platforms.  

    We found that students are using external educational websites such as Khan Academy, PatrickJMT, PurpleMath, and YouTube.  The data suggest online students use these services more than on-campus students.
  \item The use of online homework (such as WeBWorK,  MyMathLab, MyStatLab, and ALEKS) has grown significantly over the past few years. However, the data suggests that significantly more full-time instructors than part-time instructors are directing their students towards these tools (as either a required or optional component of the course).  Additionally, there is a general trend that online homework programs are being used more frequently in online classes than in on-campus classes.  Both of these discrepancies may reflect the need to distribute more information to faculty about these software resources.
  \item The math SAC needs to address whether or not we should be requiring students to use online resources that impose additional costs upon the students and, if so, what would constitute a reasonable cost to the student.  To that end, our survey asked if students would be willing to pay up to \$35 to access online homework and other resources.   We found that online students were more willing to pay an extra fee than those enrolled in on-campus classes. 
  \item The PCC mathematics website offers a wealth of materials that are frequently accessed by students. These include course-specific supplements, calculator manuals, and the required Calculus I lab manual; all of these materials were written by PCC mathematics faculty.  Students may print these materials for free from any PCC computer lab. The website also links to PCC-specific information relevant to mathematics students (such as tutoring resources) as well as outside resources (such as the Texas Instruments website).  
    \setcounter{enumi}{6}
  \item The PCC mathematics website offers a wealth of materials that are frequently accessed by students. These include course-specific supplements, calculator manuals, and the required Calculus I lab manual; all of these materials were written by PCC mathematics faculty.  Students may print these materials for free from any PCC computer lab. The website also links to PCC-specific information relevant to mathematics students (such as tutoring resources) as well as outside resources (such as the Texas Instruments website).  
    \setcounter{enumi}{8}
  \item In addition to the previously mentioned resources we also encourage students to use resources offered at PCC such as on-campus Student Learning Centers, online tutoring, Collaborate, and/or Elluminate. A significant number of students registered in on-campus sections are using these resources whereas students enrolled in online sections generally are not.  This is not especially surprising since on-campus students are, well, on campus whereas many online students rarely visit a campus . 
\end{enumerate}
\subsection{Recommendations}
The majority of our data suggests that students are using a variety of resources to further their knowledge. We recommend that instructors continue to educate students about both PCC resources and non-PCC resources. We need to uniformly encourage students to use resources such as online tutoring, student learning centers, Collaborate, and/or Elluminate; this includes resource citations in each and every course syllabus.

A broader education campaign should be engaged to distribute information to part-time faculty regarding online homework such as WeBWorK, MyMathLab, MyStatLab, and ALEKS. 

Instructors should consider quality, accessibility and cost to students when requiring specific curriculum materials. 

\section{Provide information on clerical, technical, administrative and/or tutoring support.}
