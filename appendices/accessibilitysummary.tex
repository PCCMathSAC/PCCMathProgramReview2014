% arara: pdflatex: {files: [MathSACpr2014]}
% !arara: indent: {overwrite: yes}
\chapter{Accessibility study summary}\label{app:sec:accessibility}

At the start of Fall Term 2011, PCC began its push to make online courses accessible. 
Realizing the complexity of this issue in relation to our courses in particular, the Math SAC formed a
committee to begin investigating methods for making content in online math courses accessible. 
After a few weeks of meetings and some initial experiments, the committee realized the scope,
complexity, and importance of this issue was beyond what we could do outside our regular
obligations as instructors.  Towards the end of Fall Term 2011 we submitted a request to
administration to provide two instructors with release from teaching one class for two
terms to more thoroughly investigate the topic.

Shortly before the start of Fall Term 2012, we were informed that through a combined
effort of funding, administration had granted a 1-class release for one instructor for
two terms.  Committee members, while appreciative of the offer, were concerned that this
project would weight too heavily on the shoulders of one instructor.  It would not only be overwhelming for that
instructor, but would also not allow the topic to be fully investigated.  Having two
instructors with varying backgrounds (Mac vs PC, Word vs. LaTeX, etc.), we felt the
topic could be approached from multiple angles-- a collaborative project would
be much more successful than a solo project.

As such, we requested that instead of one instructor having a one-class release for two term, we would prefer to
have two instructors to have a one-class release for one term.  This would allow for the
collaboration between two complementary math faculty members as well as spread the cost of
the project between a greater number of budgets.  The administration agreed to the revised
project and Chris Hughes and Scot Leavitt both received a one-class release for Fall Term 2012. 
Chris and Scot met with Karen Sorensen (accessibility advocate for online classes) and Andy
Freed (Manager of Technology and Support) shortly before the start of Fall Term 2012.

The initial phase of the project reoriented Scot and Chris to where they had left off
from the previous year: to build off of that work, and see what technological advances
had been achieved. They also realized that as they themselves were not
end users of assistive technologies, they needed to meet or work with people who were; this
follows the mantra "Nothing For Us Without Us."  About a third of the way through the
term, Kaela Parks introduced them to Maurice Mines, a gentleman from Washington state who is
blind and has a bit of both a technological and education background.  After the first
meeting with Maurice, it became clear that he would be a vital part of the project, and further enhanced
the collaborative nature.

Having had many successful translations of mathematical documents into various accessible
formats (printed Braille, electronic Braille file for a refreshable Braille device, webpage
for a screen reader) and having successfully printed embossed/raised graphs, Maurice agreed
to help Chris and Scot with an experiment.  They prepared a sample lecture related to a MTH 60 topic (the
slope of a line) and presented the material to Maurice in four formats: verbal presentation
with the raised graphs, as a webpage that made use of JAWS (a PC-based screen reader), as
a printed Braille document, and as a electronic Braille document to be used on a refreshable Braille device. 

Prior to the experiment, Scot and Chris were under the impression that JAWS was THE solution to
making the content in a math course accessible.  Through this initial experiment they came to
realize several (now seemingly obvious) truths:
\begin{enumerate}
	\item  Every blind student will have his/her own preferred way of receiving the content in a
	course, just as every student has his/her own learning styles.
	\item There are various grades of Braille which impacts how the mathematics should be encoded into Braille.
	\item JAWS is one of many possible assistive technologies available and is NOT the solution.
\end{enumerate}

Through additional experiments and meetings with Maurice, they learned more than they had ever expected. 
More than just learning about the technologies out there (and what might be coming in the near future),
they developed a personal connection to the topic.  The report written at the conclusion of the project \cite{accessibilityproject}
includes both a summary of our experiences, some general best practices, as well as specific recommendations for mathematics courses.

The success of the project was based on the collaborative effort between the Math SAC, the Distance
Learning Department, the respective Division deans,  and Disability Services.  While the math faculty members took on the majority
of the work, it would not have had any success without the support of Karen Sorensen, Andy Freed, Sue Quast, Loraine Schmitt,
and Kaela Parks.  Over the remainder of the 2012-13 academic year, Chris, Scot, Karen, and Kaela
presented the work and findings at eLearning 2013 Conference in San Antonio, TX, online to OCCDLA
(Oregon Community College Distance Learning Association), and the Spring 2013 ORAHEAD Conference in Corvallis, OR. 

The experience gained in this work continues to inform decisions made within the Math SAC, especially those that concern textbook selection,
and the choice to pilot new technologies. It has further enhanced our understanding and awareness of the diverse nature of our student body at PCC.
